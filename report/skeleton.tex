%--------------------------------------------------------------------------------

% How we do it: Why we do it, What particular method we use and what logic architecture we use
\section{The methodological approach}
%---What methodology we used---
- Step by step \nl
- Phase 1 \nl
- Phase 2 \nl 
- ... \nl

%---Motivation: why do we move from DS to lightweight

%\subsection{Orchestration/Choreography (maybe Composition?) in a lightweight scenario}
%- Distributed System, why they are heavy and what are some specific problems during the deployment of a Composition(problems we won't encounter in JavaRMI, speed, check my BPEL project for more) \nl
%- Lightweight scenario: why do we move here (mobile phones, Java runs everywhere, technology not ready to run BPEL engine everywhere) \nl



%---Brief Literature review to introduce the tools Java RMI, JET, and the technique MDA--- 
%\section{Tools and Techniques}


%---Introduction to the MDA and how it basically works
\subsection{The Model Driven Architecture (MDA)}
- What it is \nl
- Why we use it, why is good (because it is generic, Architecture independent, Reuse)  \nl
- How we use it: From BPEL processes to Java RMI processes \nl
\subsection{Java RMI}
- Overview, features.
- Why? it can run everywhere, lightweight scenarios  \nl
%\subsection{Jet}
%\subsection{EMF and Open Architecture Ware (Eclipse)}
%\subsection{other...} 
%----------------------------------------------------------------------------------

%---Detailed Architecture---
%\fxnote{to be done}
\section{Detailed Architecture}
\section{Implementation}
\section{Discussion}
\section{Conclusion}
















