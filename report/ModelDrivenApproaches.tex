%---Introduction to the MDA and how it basically works
\section{Model Driven Engineering}
\label{ModelDrivenEngineering}
Model Driven Engineering (MDE) is a software development methodology which focuses on creating models rather than algorithmic concepts which are usually addressed in the classical programming approaches.
This discipline attempts an abstraction of the benefits and the features provided by the Software Engineering \cite{Marrone}, usually creating a domain specific framework for implementing new systems, based on well known and tested concepts, obtained through a careful and detailed analysis of the domain and its actors.

MDE essentially shifts the focus from the program code to models, making them the gravitational center of every MDE process \cite{Lukman08}. 

The model represents the system \cite{Papa11}

% The best known MDE initiative is the Object Management Group (OMG) initiative Model-Driven Architecture (MDA), which is a registered trademark of OMG \cite{MDE}.
% 
% It is generic and groups all of the more specific techniques to develop something using models. This something can be software, ...etc.
% The objective: define methodologies and techniques to support software development through models manipulation.


% \subsection{The Model Driven Architecture (MDA)}
% \label{MDA}
% 
% \subsubsection{Model Driven Engineering}
% \label{MDE}
% \subsubsection{Model to Text transformation (M2T)}
% \label{M2T}
% \subsubsection{Acceleo}
% \label{Accelelo}

%- What it is \nl
%- Why we use it, why is good (because it is generic, Architecture independent, Reuse)  \nl
%- How we use it: From BPEL processes to Java RMI processes \nl

 
