%---Introduction to the MDA and how it basically works
\section{Model Driven Approaches}
\label{ModelDrivenApproaches}

\subsection{The Model Driven Architecture (MDA)}
\label{MDA}

\subsubsection{Model Driven Engineering}
\label{MDE}
\subsubsection{Model to Text transformation (M2T)}
\label{M2T}
\subsubsection{Acceleo}
\label{Accelelo}

%- What it is \nl
%- Why we use it, why is good (because it is generic, Architecture independent, Reuse)  \nl
%- How we use it: From BPEL processes to Java RMI processes \nl

 
%- %\subsection{MDE: Model Driven Engineering}
% Model-driven engineering (MDE) is a software development methodology which focuses on creating models rather than computing and algorithmic concepts usually addressed in the classical programming approaches.
% This discipline attempts an abstraction of the benefits and the features provided by the Software Engineering \cite{Marrone}, usually creating a domain specific framework for implementing new systems, based on well known and tested concepts, obtained through a careful and detailled analysis of the domain and its actors.
% The best known MDE initiative is the Object Management Group (OMG) initiative Model-Driven Architecture (MDA), which is a registered trademark of OMG \cite{MDE}.